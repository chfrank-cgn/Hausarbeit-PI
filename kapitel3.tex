%
%	Theorieteil
%

\pagebreak
\section{Digitales Live Studium}

\onehalfspacing

\subsection{Traditional Distance Learning}

Traditional Distance Learning, as provided by the \href{https://www.fernuni-hagen.de/english/index.shtml}{FernUniversität in Hagen} or the \href{https://www.sgd.de/}{Studiengemeinschaft Darmstadt} would consist of correspondence courses via mail, with pre-recorded material and occasional meetings with a tutor.

While this offers maximum flexibility in learning, it is different from a traditional university setting, where you join a lecture in person. Some people thrive in this flexibility; others might find it very lonely.

\subsection{Lectures}

FOM's DLS is using a different approach. It provides all the recorded material that traditional distance learning institutions would offer, but also adds live lectures from the FOM Campus to the mix.

Lectures are not pre-recorded; they happen live and can be joined online. Students can participate in the lectures, as they would in an on-campus lecture. However, for those who were unable to attend, the lecture will be recorded and available for viewing at a later time.\footnote{See \textit{FOM (2025)}: Digitales Live-Studium. \cite{fomDls}}

DLS offers the best of both worlds: the flexibility of traditional distance learning, combined with interactive lectures and live discussions with other students.

\subsection{Exams}

Exams in DLS are also online, using a proctoring software, and students can participate from the comfort of their home. The main difference from a traditional on-campus exam is that the exam will be digital, rather than on paper, and will be taken on a computer.

\subsection{DLS Studio}

To ensure consistent quality for the live lectures and the recording, FOM has built professional TV studios on its campuses for the lecturers to use.
