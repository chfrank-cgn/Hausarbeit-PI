%
%	Einfuehrung
%

\pagebreak
\section{Introduction}

\onehalfspacing

\subsection{Secondary Education}

Secondary Education

\subsection{FOM}

FOM

\subsection{COVID-19 \& Remote Work}

COVID-19

\subsection{Long Distance Education}

Long Distance Education

\subsection{Research Question \& Method}

This paper will examine FOM's new DLS offering and compare it to competing offerings. 

To do this, we will perform a Case Study and evaluate the current DLS offering.\footnote{See \textit{McCombes, S. (2019)}: What is a Case Study. \cite{caseScribbr}}

The goal is to establish whether DLS is a successful innovation.

\subsection{Gender-neutral Pronouns}

Our society is becoming more open, inclusive, and gender-fluid, and now I think it's time to think about using gender-neutral pronouns in scientific texts, too. Two well-known researchers, Abigail C. Saguy and Juliet A. Williams, both from UCLA, propose to use the singular they/them instead: "The universal singular they is inclusive of people who identify as male, female or nonbinary."\footnote{\textit{Saguy, A. (2020)}: Why We Should All Use They/Them Pronouns. \cite{pronouns}} The aim is to support an inclusive approach in science through gender-neutral language. 

In this paper, I'll attempt to follow this suggestion and invite all my readers to do the same for future articles. Thank you!

If you're not sure about the definitions of gender and sex and how to use them, have a look at the definitions\footnote{See \textit{APA (2021)}: Definitions Related to Sexual Orientation. \cite{apaDefinitions}} by the American Psychological Association.

\subsection{Climate Emergency}

As Professor Rahmstorf puts it: "Without immediate, decisive climate protection measures, my children currently attending high school could already experience a 3-degree warmer Earth. No one can say exactly what this world would look like—it would be too far outside the entire experience of human history. But almost certainly, this earth would be full of horrors for the people who would have to experience it."\footnote{\textit{Rahmstorf, A. (2024)}: Climate and Weather at 3 Degrees More. \cite{3dgreesMore}}
