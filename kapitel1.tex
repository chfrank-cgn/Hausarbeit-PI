%
%	Einfuehrung
%

\pagebreak
\section{Introduction}

\onehalfspacing

\subsection{Higher Education}

Higher education refers to formal learning that takes place after completing secondary education (e.g., high school).

Higher education usually takes place in universities that offer undergraduate and graduate degrees. Universities emphasize research and academic study and grant bachelor's, master's, and doctoral degrees.

A University of Applied Sciences (UAS) is a type of higher education institution that emphasizes practical, career-oriented education rather than theoretical research. It's also known by various names globally, e.g., Fachhochschule in Germany and Austria.

\subsection{FOM}

Here's how our university, the FOM Hochschule für Oekonomie und Management, defines itself and its target audience:

"Initiated by a non-profit foundation, FOM University has a clear educational mission: to create high-quality and affordable study programs for working professionals, trainees, high school graduates, and international students.

Since its foundation, FOM University has maintained close contacts with companies, local authorities, and associations. This strong economic and practical focus plays a key role in the transfer of knowledge. Our approximately 2,000 lecturers draw on their own business experience, teach scientific theory using practical examples, and also incorporate the perspective of working students into their teaching."\footnote{\textit{FOM (2024)}: Das besondere Format der FOM Hochschule. \cite{fomStift}}

\subsection{COVID-19 \& Remote Work}

COVID-19 fostered remote work through several key factors. Physical distancing requirements made traditional offices unsafe. Lockdowns and quarantines forced businesses to operate remotely or shut down. Companies needed to protect employee health while maintaining operations

The lockdown led to a widespread adoption of video conferencing tools (Zoom, Teams, etc.), and many companies discovered they could maintain productivity remotely. Many businesses even saw reduced overhead costs (office rent, utilities).

During COVID-19, remote work became socially normalized rather than seen as unusual, and for some employees, it became the favorite way of working.

\subsection{Long Distance Education}

Long-distance education is a method of teaching and learning where students and instructors are physically separated, using technology and various media to deliver educational content.

It can be synchronous, with live virtual classes at scheduled times, or asynchronous with self-paced study through recorded materials.

The pioneer of distance learning in higher education is the FernUniversität in Hagen: "The FernUniversität in Hagen holds a unique position as Germany's only state distance-learning university. For over 40 years, it has made higher education accessible to students who want an accredited university education but cannot or do not want to enroll in a traditional on-campus university."\footnote{\textit{FeU Hagen (2025)}: FernUniversität in Hagen. \cite{feuHagen}}

From a niche position, distance learning advanced during COVID-19 to a mainstay in higher education and prompted many established universities to evaluate remote learning opportunities.

Within FOM, this is called the \href{https://www.fom.de/digital.html}{Digitales Live Studium} and the subject of this paper.

\subsection{Research Question \& Method}

This paper will examine FOM's DLS offering and compare it to competing existing and new offerings.\footnote{See \textit{McCombes, S. (2019)}: What is a Case Study. \cite{caseScribbr}}

The goal of the paper is to establish whether DLS is a successful business model innovation.

The paper will also try to establish whether DLS is sustainable in the long run and might help FOM to become more competitive.

\subsection{Gender-neutral Pronouns}

Our society is becoming more open, inclusive, and gender-fluid, and now I think it's time to think about using gender-neutral pronouns in scientific texts, too. Two well-known researchers, Abigail C. Saguy and Juliet A. Williams, both from UCLA, propose to use the singular they/them instead: "The universal singular they is inclusive of people who identify as male, female, or nonbinary."\footnote{\textit{Saguy, A. (2020)}: Why We Should All Use They/Them Pronouns. \cite{pronouns}} The aim is to support an inclusive approach in science through gender-neutral language. 

In this paper, I'll attempt to follow this suggestion and invite all my readers to do the same for future articles. Thank you!

If you're unsure about the definitions of gender and sex and how to use them, refer to the definitions\footnote{See \textit{APA (2021)}: Definitions Related to Sexual Orientation. \cite{apaDefinitions}} by the American Psychological Association.

\subsection{Climate Emergency}

As Professor Rahmstorf puts it: "Without immediate, decisive climate protection measures, my children currently attending high school could already experience a 3-degree warmer Earth. No one can say exactly what this world would look like—it would be too far outside the entire experience of human history. But almost certainly, this earth would be full of horrors for the people who would have to experience it."\footnote{\textit{Rahmstorf, A. (2024)}: Climate and Weather at 3 Degrees More. \cite{3dgreesMore}}
