%
%	Praxisbezug
%

\pagebreak
\section{Market Situation}

\onehalfspacing

\subsection{Enrollment Figures}

With around 45,000 students, FOM is one of the largest universities in Europe.

In Germany, there are 2,900,000 students in total, according to the Federal Statistics Office.\footnote{See \textit{CHE (2025)}: Students in Germany. \cite{cheDaten}}

The majority of students in Germany are currently studying at a full university; approximately a third of all students in Germany are enrolled at a University of Applied Sciences (UAS).

Even though the number of enrollments is stagnating, there is still significant potential for growth in the market for higher education in Germany.

\subsection{Competition}

A large competitor is the IU Internationale Hochschule, with 125,000 students, which is almost triple the size of FOM. In addition to traditional distance learning, IU also offers a virtual campus now in its myStudium offering that is somewhat comparable to FOM's DLS offering.\footnote{See \textit{IU (2025)}: myStudium Studienmodell. \cite{myStudium}}

Another competitor, the Hochschule Fresenius, does not offer a virtual campus, only distance learning. Likewise, the FernUniversität in Hagen only offers distance learning.

This leaves FOM with a genuinely unique selling point of the DLS, significantly ahead of the competition.

\subsection{Hybrid Models}

While the DLS is entirely online, the in-person Campus+ model also has virtual lectures, creating a lot of synergy between the models.\footnote{See \textit{FOM (2025)}: Campus-Studium. \cite{fomCampus}}

For the virtual lectures in DLS and part of Campus+, there is no need for the lectures to be at a particular location anymore. This allows FOM to offer more subjects at different locations; however, it also enables FOM to reduce the number of lecturers and thus increase its profitability.

\subsection{Business Innovation}

With the very unique value proposition of live university lectures without the need to travel to a university campus, FOM is now ahead of its competition with a very innovative offering.

And it's not only the need for travel. In Generation Z and Alpha, the number of people with social anxiety has dramatically increased due to the multiple crises, wars, and a very bleak outlook for the future.

DLS and Campus+ both offer easier access to live lectures than traditional in-person campuses and are more suitable for neurodivergent students.

\subsection{Conclusion}

When COVID-19 turned the traditional setting of in-person work and in-person lectures on its head, FOM not only enabled remote teaching, as most schools did. FOM embraced the change as an opportunity to redefine its delivery model and create a new branch, the DLS, to combat declining enrollment rates and reach new students through a new and innovative offering.
